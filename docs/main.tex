% Compiler parameters

\documentclass[12pt, russian]{extarticle}
\usepackage[utf8]{inputenc}
\usepackage[T2A]{fontenc}
\usepackage{fontspec}
\defaultfontfeatures{Ligatures={TeX},Renderer=Basic}
\setmainfont[Ligatures={TeX,Historic}]{Times New Roman}
\usepackage[a4paper,
left=25mm,
right=15mm,
top=20mm,
bottom=20mm]{geometry}
\usepackage[linktoc=all]{hyperref}
\usepackage{titlesec}
\titlelabel{\thetitle.\quad}
\usepackage{tocloft}

\renewcommand{\cftsecfont}{}% Remove \bfseries from section titles in ToC
\renewcommand{\cftsecpagefont}{}% Remove \bfseries from section titles' page in ToC
\renewcommand{\cftsecaftersnum}{.}%

\usepackage{titlesec}
\usepackage{todonotes}

\titleformat*{\section}{\large\bfseries} % Здесь можно указать желаемый размер шрифта для \section
\titleformat*{\subsection}{\large\bfseries} % Здесь можно указать желаемый размер шрифта для \subsection
\titleformat*{\subsubsection}{\large\bfseries}

\setlength{\parindent}{1.25cm}
\setlength{\parskip}{0.4cm}
\font\subtitlefont=cmr12 at 12pt
\font\titlefont=cmr12 at 24pt
\usepackage{color}
\usepackage{mathtools}
\usepackage{listings}
\usepackage{graphicx}
\usepackage{tocloft}
\usepackage{indentfirst}
\usepackage{enumitem}
\usepackage{graphicx}
\usepackage{subcaption}
\usepackage{babel}
\usepackage{setspace}
\renewcommand{\contentsname}{}
\renewcommand{\cftsecleader}{\cftdotfill{\cftdotsep}}
\graphicspath{ {./resources/} }
\usepackage{listings}

\definecolor{dkgreen}{rgb}{0,0.6,0}
\definecolor{gray}{rgb}{0.5,0.5,0.5}
\definecolor{mauve}{rgb}{0.58,0,0.82}
\lstset{frame=none,
	language=C++,
	aboveskip=3mm,
	belowskip=3mm,
	showstringspaces=false,
	columns=flexible,
	basicstyle={\small\ttfamily},
	numbers=none,
	numberstyle=\tiny\color{gray},
	keywordstyle=\color{blue},
	commentstyle=\color{dkgreen},
	stringstyle=\color{mauve},
	breaklines=true,
	breakatwhitespace=true,
	tabsize=4
}

% End of compiler parameters

\title{}
\author{}
\date{}

\begin{document}

    \begin{titlepage}

        \begin{center}
            МИНИСТЕРСТВО ОБРАЗОВАНИЯ И НАУКИ РОССИЙСКОЙ ФЕДЕРАЦИИ \\
            Федеральное государственное автономное образовательное \\
            учреждение высшего образования \\
            «Национальный исследовательский Нижегородский государственный университет им. Н.И. Лобачевского»\\
        \end{center}

        \bigbreak

        \begin{center}
            Институт информационных технологий, математики и механики \\
            {\bfseries Кафедра Математического обеспечения и суперкомпьютерных технологий} \\
            Направление подготовки ``Инженерия программного обеспечения"
        \end{center}

        \vspace{2em}

        \begin{center}
            ОТЧЁТ \\ по учебной практике на тему \\
            {\bfseries ``Разработка программно-аппаратного комплекса для мониторинга показателей сердца
            человека''}
        \end{center}

        \vspace{5em}

        \begin{flushright}
            {\bfseries Выполнил:} \\студент группы 382008-1\\Булгаков Д.Э.\\
            \hfill Подпись \hspace{5em} \newline \\
            {\bfseries Проверил:} \\к.т.н., доц.\\ Борисов Н.А. \\
            \hfill Подпись \hspace{5em} \newline \\
        \end{flushright}


        \vspace{\fill}

        \begin{center}
            Нижний Новгород\\2024
        \end{center}
    \end{titlepage}

    \begin{spacing}{1.5}

    % Содержание

    \tableofcontents
    \thispagestyle{empty}
    \newpage

    \pagestyle{plain}
    \setcounter{page}{3}

    % Введение
    \newpage
    \section{Введение}

    Заболевания сердца и сосудов, ставшие ведущей причиной смертности по всему миру,
    находятся в центре внимания медицинского сообщества. Ишемическая болезнь сердца,
    артериальная гипертензия и другие патологии сердечно-сосудистой системы требуют
    серьезного подхода к диагностике и профилактике.

    Сердечно-сосудистая система, сложная в своей структуре и функциональности,
    подвергается различным воздействиям, которые могут привести к серьезным нарушениям.
    В этом контексте регулярные обследования приобретают ключевое значение. Они не только
    предоставляют возможность выявления начальных стадий заболеваний до появления явных
    симптомов, но и открывают перспективы для раннего вмешательства и эффективной
    профилактики. Такой подход становится неотъемлемой частью стратегии поддержания
    здоровья сердечно-сосудистой системы в условиях современного образа жизни.

    % Постановка задачи
    \newpage
    \section{Постановка задачи.}

    \begin{enumerate}
        \item \textbf{Выбрать стек технологий, на основе которых будет написан back-end сервера.}
            \begin{itemize}
                \item Определить язык программирования для сервера.
                \item Определить какую библиотеку использовать для написания сервера.
                \item Выбрать подходящую базу данных.
            \end{itemize}
        \item \textbf{Определить сценарии использования сервера.} \\
            Сценарии использования разделены на следующие платформы:
            \begin{itemize}
                \item Web сайт на основе Vue.js
                \item Electron приложение.
            \end{itemize}
            Общие сценарии можно обобщить и выделить в отдельную компоненту.
        \item \textbf{Разработать архитектуру сервера.}
        \item \textbf{Реализовать архитектуру.}
    \end{enumerate}

    \newpage
    \section{Проведенная работа.}
    \subsection{Стек технологий.}

    Перед началом проектирования архитектуры сервера, требуется выбрать язык программирования
    и фреймворк. На рассмотрении были следующие варианты:

    \begin{itemize}
        \item Node.js с фреймворком Express.js
        \item Golang с фреймворками Gin или Martini
        \item Python с фреймворками Flask или Django
    \end{itemize}

    Необходимо сравнить основные характеристики каждого решения,
    такие как производительность, удобство и простота разработки, доступность библиотек и инструментов,
    а также сообщество и поддержка. Прежде всего, было решено остановиться на языке программирования
    Python по следующим причинам:

    \begin{itemize}
        \item \textbf{Простота и читаемость кода:} \\
            Python известен своим чистым и выразительным синтаксисом,
            который делает код легким для чтения, понимания и поддержки.
        \item \textbf{Широкие возможности для веб-разработки:} \\
            Python имеет обширную экосистему библиотек и фреймворков для веб-разработки.
            Flask, Django, FastAPI - все они предоставляют мощные инструменты для создания веб-приложений
            любого уровня сложности.
        \item \textbf{Кросс-платформенность и портативность:} \\
            Python работает на множестве операционных систем, что обеспечивает гибкость и
            портативность разработки. Это означает, что код, написанный на Python,
            может быть легко перенесен с одной платформы на другую без необходимости
            внесения значительных изменений.
        \item \textbf{Богатая стандартная библиотека:} \\
            Python поставляется с обширной стандартной библиотекой, которая включает
            в себя множество модулей для работы с сетью, обработки данных, взаимодействия
            с базами данных и многое другое.
    \end{itemize}

    Осталось определиться с фреймворком Python. Было решено использовать Flask, т.к.
    сервер является учебным, небольшим проектом, он не требует таких глобальных и серьезных
    инструментов, которые предоставляет Django. Также Flask обладает более интуитивно понятный интерфейсом,
    что позволит меньше времени потратить на изучение инструмента.

    Также необходимо выбрать базу данных, которая лучше всего подойдет к выбранному фреймворку.
    На рассмотрении были следующие варианты:

    \begin{itemize}
        \item PostgreSQL
        \item SQLite
        \item MySQL
        \item MongoDB
        \item Redis
    \end{itemize}

    По итогу, было решено использовать PostgreSQL из-за его надежности, расширяемости, а также доступности.
    Кроме того, PostgreSQL активно развивается сообществом и имеет обширную документацию,
    что делает его привлекательным выбором для учебных проектов.

    \newpage
    \subsection{Сценарии использования.}

    

    \newpage
    \subsection{Разработка и реализация архитектуры.}

    \newpage
    \section{Заключение.}

    \newpage
    \section{Список литературы.}

    \newpage
    \section{Приложение.}

    \end{spacing}
\end{document}

% =============================================================================
% Compile parameters
% =============================================================================

\documentclass[12pt, russian]{extarticle}
\usepackage[utf8]{inputenc}
\usepackage[T2A]{fontenc}
\usepackage{fontspec}
\defaultfontfeatures{Ligatures={TeX},Renderer=Basic}
\setmainfont[Ligatures={TeX,Historic}]{Times New Roman}
\usepackage[a4paper,
left=25mm,
right=15mm,
top=20mm,
bottom=20mm]{geometry}
\usepackage[linktoc=all]{hyperref}
\usepackage{titlesec}
\titlelabel{\thetitle.\quad}
\usepackage{tocloft}

% Remove \bfseries from section titles in ToC
\renewcommand{\cftsecfont}{}
% Remove \bfseries from section titles' page in ToC
\renewcommand{\cftsecpagefont}{}
\renewcommand{\cftsecaftersnum}{.}
\usepackage{titlesec}

% Change font size for types
\titleformat*{\section}{\large\bfseries}
\titleformat*{\subsection}{\large\bfseries}
\titleformat*{\subsubsection}{\large\bfseries}

\setlength{\parindent}{1.25cm}
\setlength{\parskip}{0.4cm}
\font\subtitlefont=cmr12 at 12pt
\font\titlefont=cmr12 at 24pt
\usepackage{color}
\usepackage{mathtools}
\usepackage{listings}
\usepackage{graphicx}
\usepackage{tocloft}
\usepackage{indentfirst}
\usepackage{enumitem}
\usepackage{graphicx}
\usepackage{subcaption}
\usepackage{babel}
\usepackage{setspace}
\renewcommand{\contentsname}{}
\renewcommand{\cftsecleader}{\cftdotfill{\cftdotsep}}
\graphicspath{ {./resources/} }

\usepackage{listings}
\definecolor{dkgreen}{rgb}{0,0.6,0}
\definecolor{gray}{rgb}{0.5,0.5,0.5}
\definecolor{mauve}{rgb}{0.58,0,0.82}
\lstset{
    language=Python,
    basicstyle=\ttfamily\small,
    keywordstyle=\color{blue},
    commentstyle=\color{dkgreen},
    stringstyle=\color{mauve},
    stepnumber=1,
    breaklines=true,
    breakatwhitespace=true,
    tabsize=4,
    captionpos=tl,
}

% =============================================================================
% End of compile parameters
% =============================================================================

\title{}
\author{}
\date{}

\begin{document}

% =============================================================================
% Global titlepage
% =============================================================================

\begin{titlepage}

    \begin{center}
        МИНИСТЕРСТВО НАУКИ И ВЫСШЕГО ОБРАЗОВАНИЯ РОССИЙСКОЙ ФЕДЕРАЦИИ \\
        Федеральное государственное автономное образовательное учреждение \\
        высшего образования \\
        \textbf{
            «Национальный исследовательский \\
            Нижегородский государственный университет им. Н.И. Лобачевского»\\ (ННГУ)
        }
    \bigbreak

    \vspace{2em}
        \textbf{
            Институт информационных технологий, математики и механики
            \bigbreak
            Кафедра математического обеспечения и суперкомпьютерных технологий
        }

        Направление подготовки: «Программная инженерия» \\
        Профиль подготовки: «Разработка программно-информационных систем»

        \bigbreak
        \bigbreak
        \bigbreak

        \textbf{ВЫПУСКНАЯ КВАЛИФИКАЦИОННАЯ РАБОТА БАКАЛАВРА}
        \bigbreak

        на тему \\
        {\bfseries ``Разработка программно-аппаратного комплекса для мониторинга показателей сердца
        человека''}
    \end{center}

    \vspace{5em}

    \begin{flushright}
        {\bfseries Выполнил:} студент группы \\ 382008-1 Булгаков Даниил Эдуардович\\
        \hfill Подпись \hspace{5em} \newline \\
        {\bfseries Научный руководитель:} \\доцент кафедры МОСТ, к.т.н., \\ Борисов Николай Анатольевич \\
        \hfill Подпись \hspace{5em} \newline \\
    \end{flushright}


    \vspace{\fill}

    \begin{center}
        Нижний Новгород\\2024
    \end{center}

\end{titlepage}

% =============================================================================
% Main content
% =============================================================================

% ========== Set global spacing ==========
\begin{spacing}{1.5}

% ========== Table of content ==========
\tableofcontents
\thispagestyle{empty}
\newpage

% Params to make the following text start with
% its page number
\pagestyle{plain}
\setcounter{page}{3}

% ========== Introduction ==========
\section{Введение}

В последние десятилетия наблюдается значительный рост числа заболеваний сердечно-сосудистой системы, что делает мониторинг состояния сердца важной задачей в области медицины. Одним из наиболее распространенных методов диагностики и наблюдения за состоянием сердца является электрокардиография (ЭКГ). ЭКГ представляет собой графическую запись электрической активности сердца, которая позволяет выявлять различные аномалии, такие как аритмии, ишемия, инфаркты и другие патологии. Данный метод широко применяется благодаря своей информативности, неинвазивности и доступности.

Несмотря на то, что традиционные стационарные системы ЭКГ являются высокоэффективными, их использование ограничено условиями медицинских учреждений. Пациенты, особенно те, кто страдает хроническими заболеваниями, нуждаются в постоянном мониторинге сердечной активности, что затруднительно в условиях стационара. В этой связи актуальной становится разработка портативных систем для непрерывного мониторинга показателей сердца в повседневной жизни.

Цель данной дипломной работы заключается в разработке программно-аппаратного комплекса для мониторинга показателей сердца человека. Комплекс включает в себя модуль для снятия ЭКГ, приложение для передачи данных с модуля на сервер для сбора, анализа и хранения данных. Такое решение позволяет не только повысить качество мониторинга, но и обеспечить своевременное реагирование на изменения состояния здоровья пациента.

\noindent \textbf {Актуальность и значимость проекта \\}
Разработка портативного комплекса для мониторинга ЭКГ имеет значительное практическое значение. Он позволяет:

\begin{itemize}
\item Обеспечить круглосуточное наблюдение за состоянием сердца пациентов, не ограничивая их мобильность.
\item Снизить нагрузку на медицинский персонал за счет автоматизации сбора и первичной обработки данных.
\item Повысить точность диагностики благодаря постоянному потоку данных и возможности их анализа в динамике.
\end{itemize}

% ========== Task definition ==========
\newpage
\section{Постановка задачи}

\noindent \textbf{Модуль снятия ЭКГ.}

\begin{enumerate}

    \item \textbf{Аппаратная часть.}
    \begin{itemize}
        \item Выбор технологий передачи данных.
        \item Выбор модуля снятия ЭКГ.
        \item Выбор микроконтроллера.
        \item Разработка 3D модели.
        \item Создания схемы подключения всех компонентов.
    \end{itemize}

    \item \textbf{Программная часть.}
    \begin{itemize}
        \item Выбор протокола передачи данных на приложение.
        \item Выбор фреймворка и языка программирования микроконтроллера.
    \end{itemize}

\end{enumerate}

\noindent \textbf{Компьютерное приложение.}

\begin{itemize}
    \item Выбор фреймворка для разработки приложения.
    \item Реализовать передачу данных с компьютера на сервер.
    \item Включить в приложение механизм авторизации для обеспечения безопасности передаваемых данных.
\end{itemize}

\noindent \textbf{Веб-приложение.}

\begin{enumerate}

    \item \textbf{Frontend.}
    \begin{itemize}
        \item Выбор фреймворка.
        \item Разработка интуитивно понятного пользовательского интерфейса \\
            для конечных пользователей.
    \end{itemize}

    \item \textbf{Backend.}
    \begin{itemize}
        \item Выбор фреймворка и систему управления реляционными базами данных.
        \item Реализовать API для взаимодействия с Fronted частью веб-приложения.
        \item Обеспечить меры безопасности для передачи и хранения медицинских данных.
    \end{itemize}

\end{enumerate}

% ========== Work made ==========
\newpage
\section{Проведенная работа}

% ========== Conclusion ==========
\newpage
\section{Заключение}

% ========== References ==========
\newpage
\section{Список литературы}

% ========== Additional 
\newpage
\section{Приложение}

\end{spacing}
\end{document}

% =============================================================================
% Compile parameters
% =============================================================================

\documentclass[12pt, russian]{extarticle}
\usepackage[utf8]{inputenc}
\usepackage[T2A]{fontenc}
\usepackage{fontspec}
\defaultfontfeatures{Ligatures={TeX},Renderer=Basic}
\setmainfont[Ligatures={TeX,Historic}]{Times New Roman}
\usepackage[a4paper,
left=25mm,
right=15mm,
top=20mm,
bottom=20mm]{geometry}
\usepackage[linktoc=all]{hyperref}
\usepackage{titlesec}
\titlelabel{\thetitle.\quad}
\usepackage{tocloft}

% Remove \bfseries from section titles in ToC
\renewcommand{\cftsecfont}{}
% Remove \bfseries from section titles' page in ToC
\renewcommand{\cftsecpagefont}{}
\renewcommand{\cftsecaftersnum}{.}
\usepackage{titlesec}

% Change font size for types
\titleformat*{\section}{\large\bfseries}
\titleformat*{\subsection}{\large\bfseries}
\titleformat*{\subsubsection}{\large\bfseries}

\setlength{\parindent}{1.25cm}
\setlength{\parskip}{0.4cm}
\font\subtitlefont=cmr12 at 12pt
\font\titlefont=cmr12 at 24pt
\usepackage{color}
\usepackage{mathtools}
\usepackage{listings}
\usepackage{graphicx}
\usepackage{tocloft}
\usepackage{indentfirst}
\usepackage{enumitem}
\usepackage{graphicx}
\usepackage{subcaption}
\usepackage{babel}
\usepackage{setspace}
\renewcommand{\contentsname}{}
\renewcommand{\cftsecleader}{\cftdotfill{\cftdotsep}}
\graphicspath{ {./resources/} }

\usepackage{listings}
\definecolor{dkgreen}{rgb}{0,0.6,0}
\definecolor{gray}{rgb}{0.5,0.5,0.5}
\definecolor{mauve}{rgb}{0.58,0,0.82}
\lstset{
    language=Python,
    basicstyle=\ttfamily\small,
    keywordstyle=\color{blue},
    commentstyle=\color{dkgreen},
    stringstyle=\color{mauve},
    stepnumber=1,
    breaklines=true,
    breakatwhitespace=true,
    tabsize=4,
    captionpos=tl,
}

% =============================================================================
% End of compile parameters
% =============================================================================

\title{}
\author{}
\date{}

\begin{document}

% =============================================================================
% Global titlepage
% =============================================================================

\begin{titlepage}

    \begin{center}
        МИНИСТЕРСТВО НАУКИ И ВЫСШЕГО ОБРАЗОВАНИЯ РОССИЙСКОЙ ФЕДЕРАЦИИ \\
        Федеральное государственное автономное образовательное учреждение \\
        высшего образования \\
        \textbf{
            «Национальный исследовательский \\
            Нижегородский государственный университет им. Н.И. Лобачевского»\\ (ННГУ)
        }
    \bigbreak

    \vspace{2em}
        \textbf{
            Институт информационных технологий, математики и механики
            \bigbreak
            Кафедра математического обеспечения и суперкомпьютерных технологий
        }

        Направление подготовки: «Программная инженерия» \\
        Профиль подготовки: «Разработка программно-информационных систем»

        \bigbreak
        \bigbreak
        \bigbreak

        \textbf{ВЫПУСКНАЯ КВАЛИФИКАЦИОННАЯ РАБОТА БАКАЛАВРА}
        \bigbreak

        на тему \\
        {\bfseries ``Разработка программно-аппаратного комплекса для мониторинга показателей сердца
        человека''}
    \end{center}

    \vspace{5em}

    \begin{flushright}
        {\bfseries Выполнил:} студент группы \\ 382008-1 Булгаков Даниил Эдуардович\\
        \hfill Подпись \hspace{5em} \newline \\
        {\bfseries Научный руководитель:} \\доцент кафедры МОСТ, к.т.н., \\ Борисов Николай Анатольевич \\
        \hfill Подпись \hspace{5em} \newline \\
    \end{flushright}


    \vspace{\fill}

    \begin{center}
        Нижний Новгород\\2024
    \end{center}

\end{titlepage}

% =============================================================================
% Main content
% =============================================================================

% ========== Set global spacing ==========
\begin{spacing}{1.5}

% ========== Table of content ==========
\tableofcontents
\thispagestyle{empty}
\newpage

% Params to make the following text start with
% its page number
\pagestyle{plain}
\setcounter{page}{3}

% ========== Introduction ==========
\section{Введение}

В последние десятилетия наблюдается значительный рост числа заболеваний сердечно-сосудистой системы, что делает мониторинг состояния сердца важной задачей в области медицины. Одним из наиболее распространенных методов диагностики и наблюдения за состоянием сердца является электрокардиография (ЭКГ). ЭКГ представляет собой графическую запись электрической активности сердца, которая позволяет выявлять различные аномалии, такие как аритмии, ишемия, инфаркты и другие патологии. Данный метод широко применяется благодаря своей информативности, неинвазивности и доступности.

Несмотря на то, что традиционные стационарные системы ЭКГ являются высокоэффективными, их использование ограничено условиями медицинских учреждений. Пациенты, особенно те, кто страдает хроническими заболеваниями, нуждаются в постоянном мониторинге сердечной активности, что затруднительно в условиях стационара. В этой связи актуальной становится разработка портативных систем для непрерывного мониторинга показателей сердца в повседневной жизни.

Цель данной дипломной работы заключается в разработке программно-аппаратного комплекса для мониторинга показателей сердца человека. Комплекс включает в себя модуль для снятия ЭКГ, приложение для передачи данных с модуля на сервер сбора, анализа и хранения данных.

% Разработка портативного комплекса для мониторинга ЭКГ имеет значительное практическое значение. Он позволяет:

% \begin{itemize}
% \item Обеспечить круглосуточное наблюдение за состоянием сердца пациентов, не ограничивая их мобильность.
% \item Снизить нагрузку на медицинский персонал за счет автоматизации сбора и первичной обработки данных.
% \item Повысить точность диагностики благодаря постоянному потоку данных и возможности их анализа в динамике.
% \end{itemize}

% ========== Task definition ==========
\newpage
\section{Описание предметной области}

\subsection{ЭКГ}

Электрокардиография — методика регистрации и исследования электрических полей, образующихся при работе сердца. Для получения значения разности потенциалов на участке тела человека используются электроды. Набор электродов, расположенных в определенных местах, формирует различные виды отведений:

\begin{itemize}
    \item Стандартные отведения
    \item Усиленные отведения
    \item Грудные отведения
\end{itemize}

\subsubsection{Стандартные отведения}

Стандартные отведения регистрируют разность потенциалов между конечностями человека. Для получения данного типа отведений требуется три электрода: положительный, отрицательный и заземление. Так, правая и левая пара электродов руки образуют первое стандартное отведение - I, электроды правой руки и левой ноги – второе - II, третье отведение III - левая рука и левая нога. Третий электрод используется как заземление (Рис. 1).

\begin{figure}[htbp]
\centering
\includegraphics[scale=0.44]{resources/отведения экг.jpg}
\caption{Стандартные отведения.}
\end{figure}

Тогда для получения кардиограммы достаточно вычислить разность потенциалов между указанными сигналами.

\begin{center}
\begin{tabular}{|l|l|}
\hline
\textbf{Отведение}      & \textbf{Вычисление} \\ \hline
\textbf{1-ое отведение} & LA-RA               \\ \hline
\textbf{2-ое отведение} & LL-RA               \\ \hline
\textbf{3-е отведение}  & LL-LA               \\ \hline
\end{tabular} \bigbreak
LA - левая рука, RA - правая рука, LL - левая нога.
\end{center}

Нетрудно заметить, что в случае, когда нам требуется получить значения сразу по трем отведениям, то аппаратно потребуется считывать только два из них, так как третье можно вычислить путем сложения/вычитание двух других, к примеру:

\begin{center}
    1-ое отведение + 3-е отведение = 2-е отведение
\end{center}

Данные отведения позволяют регистрировать следующие типы заболеваний: 

\begin{itemize}
    \item Ишемия миокарда (недостаточное поступление кислорода в сердечную мышцу). Это может проявляться в виде изменений в зубцах ST и T, а также снижения амплитуды зубцов.
    \item Аритмии, такие как фибрилляция предсердий или желудочковые экстрасистолы. Это может проявляться в виде изменений в ритме и частоте сердечных сокращений, а также в форме зубцов на ЭКГ.
    \item Блокады проводимости сердца, такие как блокада правой ножки пучка Гиса. Это может проявляться в виде изменений в продолжительности и форме зубцов на ЭКГ.
\end{itemize}

\subsubsection{Усиленные отведения}

Усиленные отведения по принципу очень схожи со стандартными отведениями, для них также требуется три электрода. Однако они регистрируют разность потенциалов между одной из конечностей, на которой помещён активный положительный электрод данного отведения и суммарный электродом двух других конечностей. Существуют три таких отведения:

\begin{itemize}
    \item aVR - усиленное отведение правой руки
    \item aVL - усиленное отведение левой руки
    \item aVF - усиленное отведение левой ноги
\end{itemize}

Для вычисления можно использовать как сигналы с конечностей, так и значения стандартных отведений, используя таблицу.

\begin{center}
\begin{tabular}{|l|l|l|}
\hline
\textbf{Отведение} & \textbf{Вычисление} & \textbf{Аналог} \\ \hline
\textbf{aVR}       & RA-0.5*(LA+LL)      & -0.5*(I+II)     \\ \hline
\textbf{aVL}       & LA-0.5*(LL+RA)      & 0.5*(I-III)     \\ \hline
\textbf{aVF}       & LL-0.5*(LA+RA)      & 0.5*(II+III)    \\ \hline
\end{tabular} \bigbreak
LA - левая рука, RA - правая рука, LL - левая нога \\
I, II, III - типы стандартных отведений
\end{center}

Данные отведения используются для оценки электрической активности сердца в переднезаднем направлении. Они могут помочь выявить инфаркт миокарда.

\subsubsection{Грудные отведения}

Грудные отведения регистрируют разницу потенциалов между позитивным электродом, установленным в определённой точке грудной клетки (всего их 6) и единым для остальных пяти электродом Вильсона, потенциал которого равняется нулю (Рис. 2).

\begin{figure}[htbp]
\centering
\includegraphics[scale=0.46]{resources/грудные отведения .png}
\caption{Грудные отведения.}
\label{fig:my_label}
\end{figure}

Данные отведения используются для оценки электрической активности сердца в горизонтальной плоскости. Они могут помочь выявить заболевания миокарда, такие как инфаркт миокарда и аномалии развития сердца.

\newpage
\subsubsection{Вывод}

Таким образом, регистрация двух отведений ЭКГ, будь то стандартные или усиленные, является ключевым этапом в диагностике сердечных заболеваний. Этот метод позволяет получить обширную информацию о состоянии сердца пациента, включая анализ электрической активности в различных направлениях.

\newpage
\subsection{Подходы к мониторингу сердечной активности}

% Сердечно-сосудистые заболевания (ССЗ) являются одной из ведущих причин смертности во всем мире. Эффективное управление этими заболеваниями требует своевременной диагностики и постоянного мониторинга состояния сердца. 

В настоящее время существуют различные подходы к мониторингу сердечной активности, включая:

\begin{enumerate}
    \item \textbf{Традиционные стационарные ЭКГ системы:} \\
        Высокая точность и надежность, но требуют нахождения в клинике.
    \item \textbf{Портативные ЭКГ устройства:} \\
        Легкие и удобные в использовании, но часто имеют ограниченные возможности по длительности работы и качеству связи.
    \item \textbf{Носимые устройства (например, смарт-часы с функцией ЭКГ):} \\ 
        Удобны для повседневного использования, но часто менее точны и имеют ограничения по функциональности.
\end{enumerate}

Выбор портативных ЭКГ устройств обусловлен стремлением к сочетанию высокой точности измерений с максимальной мобильностью и удобством использования, что в конечном итоге позволит обеспечить непрерывный и эффективный мониторинг сердечной активности пациента в реальном времени для своевременного реагирования на изменения состояния здоровья пациента.

% ========== Work made ==========
\newpage
\section{Разработка проектного решения}

Архитектура системы играет ключевую роль в обеспечении надежного и эффективного функционирования программно-аппаратного комплекса для мониторинга показателей сердца. При её проектировании было принято решение использовать принцип многослойной архитектуры. Это решение было обосновано следующими причинами:

\begin {enumerate}
    \item \textbf{Разделение ответственностей:} \\
        Многослойная архитектура позволяет четко разделить компоненты системы на логические уровни, каждый из которых отвечает за свои специфические функции. Например, уровень сбора данных отвечает за получение и обработку информации от датчиков, уровень приложения - за визуализацию и интеракцию с пользователем, а уровень сервера - за обработку и хранение данных.
    \item \textbf{Повторное использование кода:} \\
        Благодаря модульной структуре многослойной архитектуры компоненты системы становятся более независимыми и переиспользуемыми. Это позволяет избежать дублирования кода и обеспечить более эффективную разработку и поддержку программного продукта.
    \item \textbf{Гибкость и масштабируемость:} \\
        Многослойная архитектура обеспечивает гибкость в изменении и модификации системы. Новые функциональные возможности могут быть легко добавлены или изменены без влияния на другие компоненты. Кроме того, такая архитектура легко масштабируется при необходимости увеличения производительности или добавления новых узлов.
    \item \textbf{Улучшенная поддержка и тестирование:} \\
        Четкое разделение компонентов упрощает процесс поддержки и тестирования системы. Каждый уровень может быть протестирован отдельно, что позволяет выявлять и устранять ошибки на ранних этапах разработки.
\end{enumerate}

\newpage
Архитектура проекта разделена на пять основных слоев (Рис. 3):

\begin{figure}[htbp]
    \centering
    \includegraphics[scale=0.85]{resources/arch_layers.png}
    \caption{Архитектура.}
    \label{fig:my_label}
\end{figure}


\begin {enumerate}
    \item \textbf{Модуль сбора данных (Hardware Layer):} \\
        Он включает в себя аппаратное обеспечение для снятия показаний ЭКГ, такое как микроконтроллер ESP32 и усилитель сигнала AD8232.
        Микроконтроллер ESP32 отвечает за считывание данных с электродов и преобразование их в цифровой формат.
        Усилитель сигнала AD8232 усиливает сигнал ЭКГ для более точного считывания.

    \item \textbf{Приложение для передачи данных (Application Layer):} \\
        Отвечает за прием, обработку и передачу данных с модуля сбора данных на сервер для анализа и хранения.
        В приложении реализована логика работы с данными, включая их форматирование, упаковку и отправку на сервер.

    \item \textbf{Сервер для сбора, анализа и хранения данных (Server Layer):} \\
        Этот слой осуществляет прием, анализ и хранение данных, полученных от приложения для передачи данных.
        В рамках сервера реализованы функциональности по обработке данных, анализу показателей сердечной активности и хранению результатов.

    \item \textbf{База данных (Data Storage Layer):} \\
        Обеспечивает долговременное хранение данных и предоставляет интерфейсы для их извлечения и манипуляций.
        Данные о показателях сердечной активности сохраняются в базе данных для последующего доступа и анализа.
        Для обеспечения надежности и масштабируемости, база данных может быть построена на основе реляционных или NoSQL технологий.

    \item \textbf{Web Layer (Веб-приложение):} \\
        Этот слой предоставляет пользовательский интерфейс для взаимодействия с системой через веб-браузер. Пользователи могут просматривать данные, настраивать параметры мониторинга и получать заключения по результатам исследования ЭКГ.
\end{enumerate}

% ========== Product hardware development ==========
\newpage
\subsection{Разработка аппаратной части комплекса (Hardware Level)}

В аппаратной части программно-аппаратного комплекса для мониторинга показателей сердца человека решающее значение имеет выбор компонентов, обеспечивающих сбор и передачу данных. Микроконтроллеры, сенсоры и другие устройства должны быть тщательно подобраны с учетом их функциональности, надежности и совместимости с основными целями проекта. 

В этом разделе будет рассмотрен процесс выбора и обоснование использования конкретных компонентов, фокусируясь на микроконтроллере, датчиках для снятия ЭКГ и других важных устройствах для поддержки работоспособности модуля в целом.

\subsubsection{Выбор микроконтроллера}

При выборе микроконтроллера рассматривались следующие характеристики:

\begin{itemize}
    \item Размер платы
    \item Энергопотребление
    \item Вычислительная мощность (кол-во ядер)
    \item Количество пинов
\end{itemize}

С учетом всех факторов, были выбраны микроконтроллеры ESP32 и ESP32-3C.
Ниже приведены основные сравнительные характеристики. 

\begin{center}
\begin{tabular}{|l|l|l|}
\hline
\textbf{}            & \textbf{ESP32} & \textbf{ESP32-3C} \\ \hline
\textbf{CPU}         & Xtensa LX6                             & RISC-V                                    \\ \hline
\textbf{Кол-во ядер} & 2                                      & 1                                         \\ \hline
\textbf{Пины GPIO}        & 34                                     & 22                                        \\ \hline
\textbf{Потребление} & До 325 мА                              & До 240 мА                                 \\ \hline
\textbf{Размеры}     & 31 x 18 x 3.0 мм                       & 24 x 16 x 3.1 мм                          \\ \hline
\end{tabular}
\end{center}

По итогу решено использовать микроконтроллер ESP32-3C поскольку он обладает меньшими размерами и более низким энергопотреблением, однако придется столкнуться с проблемами производительности поскольку выбранный микроконтроллер имеет одно ядро и нельзя будет отправлять полученные данные в фоновом режиме на втором ядре.

\subsubsection{Выбор датчика для снятия ЭКГ}

При выборе модели датчика сердечного ритма были определены следующие критерии, которым он должен соответствовать: 

\begin{enumerate}
    \item Датчик должен поддерживать считывание на частоте в 100 Гц
    \item Датчик не должен иметь больших размеров
    \item Измерения датчика должны быть разборчивыми
    \item Цена датчика должна быть бюджетной
\end{enumerate}

После проведения анализа было решено остановиться на следующих датчиках: 

\begin{itemize}
    \item AD620
    \item AD623
    \item AD8232
    \item CJMCU-333
\end{itemize}

Далее проходил этап их сравнения на тестовом стенде.

Тестирующий стенд был разработан для одновременного анализа данных со всех четырех датчиков.
Основным питающим элементом была литиевая батарея, чтобы избежать зашумления сигнала, которое может возникнуть при питании от сети.

Приведем описание для датчиков, которые были главными претендентами на выбор.

\subsubsection{Тестирование CJMCU-333}

CJMCU333 — маломощный прецизионный инструментальный усилитель. Датчик имеет универсальную конструкцию с тремя операционными усилителями, небольшой размер и малое энергопотребление. Один внешний резистор устанавливает любой коэффициент усиления от 1 до 1000 (Рис. 4).

\begin{figure}[htbp]
\centering
\includegraphics[scale=0.15]{resources/cjmcu333/cjmcu-333.png}
\caption{CJMCU-333}
\label{fig:my_label}
\end{figure}

После калибровки и тестирования, ЭКГ выглядело следующим образом (Рис. 5, 6).

\begin{figure}[htbp]
\centering
\includegraphics[scale=0.6]{resources/cjmcu333/1.png}
\caption{CJMCU-333 ЭКГ Пример-1.}
\label{fig:my_label}
\end{figure}

\begin{figure}[htbp]
\centering
\includegraphics[scale=0.6]{resources/cjmcu333/2.png}
\caption{CJMCU-333 ЭКГ Пример-2.}
\label{fig:my_label}
\end{figure}

\newpage


По графику невозможно отследить периодичность. Кривая имеет большое количество артефактов и шума. Разобрать данную ЭКГ-кардиограмму не представляется возможным. Результаты тестирования были неудовлетворительными, поэтому данный датчик применяться в проекте не будет.

\subsubsection{Тестирование AD8232}

AD8232 — это интегрированный блок формирования сигнала для ЭКГ и других приложений измерения биопотенциала. Он предназначен для извлечения, усиления и фильтрации слабых сигналов биопотенциала в условиях шумов, например, создаваемых движением или удаленным размещением электродов (Рис. 7).

\begin{figure}[htbp]
\centering
\includegraphics[scale=1.4, angle=90]{resources/ad8232/ad8232.png}
\caption{AD8232}
\label{fig:my_label}
\end{figure}

При тестировании график ЭКГ выглядел следующим образом (Рис. 8).

\begin{figure}[htbp]
\centering
\includegraphics[scale=0.5]{resources/ad8232/1е-отведение.jpg}
\caption{График AD8232 1-ое отведение.}
\label{fig:my_label}
\end{figure}

По результатам первых тестов можно заметить, что показатели получаются довольно точными и чистыми. Кривая имеет выраженную периодичность, а также точно прослеживаются зубцы. В качестве эксперимента было проведено считывание ЭКГ по второму отведению (Рис. 9).

\begin{figure}[htbp]
\centering
\includegraphics[scale=0.8]{resources/ad8232/2е-отведение.jpg}
\caption{График AD8232 2-ое отведение.}
\label{fig:my_label}
\end{figure}

А также по значениям первого и второго отведений было вычислено третье отведение с последующем отображением значений в виде графика (Рис. 9).

\begin{figure}[htbp]
\centering
\includegraphics[scale=0.8]{resources/ad8232/3е-отведение.jpg}
\caption{График AD8232 3-ое отведение.}
\label{fig:my_label}
\end{figure}

Таким образом, из всех протестированных датчиков был выбран именно AD8232, поскольку только с его помощью удалось достичь таких качественных результатов.

\section{Разработка модели}

При разработке модуля необходимо учитывать возможность его размещения на спортивной майке, а значит он должен иметь относительно небольшие размеры и маленький вес. Так как модуль необходимо будет заряжать и отправлять с него данные на сервер было принято решение разбить его на два подмодуля. Так, в первом подмодуле будут располагаться электроды и датчики считывания ЭКГ, а во втором то, что необходимо для сохранения, обработки и передачи полученных данных.

\par\textbf{Подмодуль ЭКГ}

Экспериментальным путем было выяснено, что наиболее подходящей моделью датчика сердечного ритма является AD8232. Так как нам требуется считывать два типа отведений минимум, а каждый датчик может считывать максимум один, нам потребуется использовать их сразу два. Примитивная схема данного подмодуля выглядит следующим образом (Рис. 11).

\begin{figure}[htbp]
\centering
\includegraphics[scale=0.5]{resources/sub1.png}
\caption{Подмодуль ЭКГ.}
\label{fig:my_label}
\end{figure}

На вход в данный подмодуль поступает четыре провода:

\begin{itemize}
    \item SDN1
    \item GND
    \item 3v3
    \item SDN2
\end{itemize}

Провода типа SDN отвечают за перевод подключенного к ним датчика в энергосберегающий режим. GND и 3v3 отвечают за питание датчиков.

На выход идет шесть проводов: 

\begin{itemize}
    \item два провода типа LO1
    \item два провода типа LO2
    \item OUT1
    \item OUT2
\end{itemize}

Провода типа LO передают сигнал о том, что электроды подключены к датчику и с них считывается сигнал. Провод OUT1 отправляет значение ЭКГ по первому типу отведения, OUT2 по второму.

\par\textbf{Подмодуль с микроконтроллером}

Основная задача данного подмодуля является считывание данных с датчиков ЭКГ, промежуточное хранение и их последующая отправка. Для сохранения большого объема информации на определенное время требуется внешний накопитель. Так как модуль является автономным, для него требуется батарея с возможностью зарядки.

Таким образом, минимальный набор требуемых элементов выглядит следующим образом:

\begin{itemize}
    \item ESP32-3C
    \item Преобразователь напряжение в 3В
    \item Модуль с SD-картой
    \item Модуль для зарядного устройства
    \item Литий-ионная батарея
    \item Магнитный коннектор
\end{itemize}

Исходя из этого, была разработана примитивная схема данного подмодуля (Рис. 12).
\newpage

\begin{figure}[htbp]
\centering
\includegraphics[scale=0.48]{resources/2.png}
\caption{Подмодуль с микроконтроллером.}
\label{fig:my_label}
\end{figure}

Входные и выходные провода соединятся с подмодулем ЭКГ, поэтому ознакомится с назначением каждого провода можно в вышеупомянутом разделе.

Для удобства была разработана тестовая 3D модель, в которую были размещены все необходимые компоненты.

% ========== Product software development ==========
\newpage
\subsection{Разработка программной части комплекса (Software level)}


В программной части программно-аппаратного комплекса для мониторинга показателей сердца человека не менее важен выбор правильных технологий и методов разработки.
Разработка приложения для передачи данных, серверной части и веб-интерфейса требует грамотного подхода и выбора наиболее подходящих инструментов. 

В данном разделе будет рассмотрен процесс выбора технологий и методов разработки, обоснование принятых решений и анализ их влияния на функциональность и эффективность всего комплекса мониторинга, а также определение функциональных и нефункциональных требований.

\subsubsection{Разработка требований к программному продукту (ПП)}

Требования к программному продукту можно разделить на функциональные и нефункциональные.

\noindent
\textbf{Функциональные требования:}

\begin{enumerate}
    \item \textbf{Сбор данных ЭКГ:} \\
        ПП должен уметь собирать данные ЭКГ с помощью модуля ESP32 и AD8232.
    \item \textbf{Передача данных:} \\
        Данные ЭКГ должны передаваться от модуля на сервер в реальном времени.
    \item \textbf{Анализ данных:} \\
        Сервер должен анализировать поступающие данные, выявлять аномалии и предупреждать пользователя.
    \item \textbf{Хранение данных:} \\
        Данные должны сохраняться в базе данных PostgreSQL для последующего анализа и отчетности.
    \item \textbf{Пользовательский интерфейс:} \\
        ПП должен иметь удобный пользовательский интерфейс для просмотра текущих и исторических данных ЭКГ, а также для получения заключений по показаниям ЭКГ.
    \item \textbf{Отображение данных в реальном времени:} \\
        Пользователям должна быть предоставлена
    возможность мониторинга показаний ЭКГ в реальном времени через веб-интерфейс.
\end{enumerate}

\noindent
\textbf{Нефункциональные требования:}

\begin{enumerate}
    \item \textbf{Отказоустойчивость:} \\
        Приложение должно быть устойчивым к сбоям и обеспечивать непрерывную работу в течение длительного времени.
    \item \textbf{Безопасность данных:} \\
        Все данные, передаваемые между компонентами системы, должны быть защищены с помощью соответствующих механизмов шифрования и аутентификации.
    \item \textbf{Производительность:} \\
        Приложение должно обеспечивать высокую производительность при передаче и обработке данных, чтобы минимизировать задержки и обеспечить оперативную реакцию на изменения состояния пациента.
    \item \textbf{Масштабируемость:} \\
        Система должна быть способна масштабироваться в зависимости от количества пользователей и объема данных, обрабатываемых ежедневно.
    \item \textbf{Простота использования:} \\
        Веб-интерфейс должен быть интуитивно понятным и легким в использовании даже для неопытных пользователей.
\end{enumerate}

% ========== Application layer ==========
\subsubsection{Код для модуля сбора данных на базе ESP32}

\par \noindent \textbf{Выбор фреймворка}

Для разработки программного обеспечения на ESP32 был выбран Arduino Framework в связке с PlatformIO благодаря простоте и удобству использования. Arduino Framework предоставляет все необходимые инструменты для написания, компиляции и загрузки кода на микроконтроллер. Его поддержка широкого спектра библиотек и большая пользовательская база делают решение возникающих проблем более легким.

\par \noindent \textbf{Архитектура}

При проектировании программного обеспечения использовался модульный подход, что позволяет изолировать различные функциональные компоненты системы. Это упрощает разработку, отладку и сопровождение кода, а также способствует повторному использованию и легкости расширения.

В системе используется паттерн Singleton, что позволяет гарантировать, что каждый модуль имеет только одну глобальную точку доступа. Это особенно важно для модулей, которые управляют аппаратными ресурсами, такими как SD-карта или BLE-сервисы. Singleton паттерн обеспечивает уникальность экземпляра и глобальный доступ к нему, что упрощает управление состоянием и ресурсами.

\par \noindent \textbf{Структура проекта}

Работа с модулями организована путем разделения на заголовочные файлы (.h) и файлы реализации (.cpp). Заголовочные файлы содержат объявления функций и классов, обеспечивая интерфейс между модулями, тогда как файлы реализации содержат конкретную реализацию этих функций.

\par \noindent \textbf{Модули и точка входа}

\begin{enumerate}
    \item \textbf{BLE Service Handler Module} \\
        Обрабатывает взаимодействие через Bluetooth Low Energy (BLE). Включает функции для настройки и обработки BLE-соединений.
    \item \textbf{Configuration Helper Module} \\
        Содержит параметры конфигурации, такие как настройки сети и параметры подключения. Обеспечивает централизованное управление конфигурациями.
    \item \textbf{Data Package Module} \\
        Осуществляет упаковку и распаковку данных для передачи. Включает структуры данных и функции для их обработки.
    \item \textbf{SPI Flash Module} \\
        Управляет взаимодействием с SPI Flash памятью. Содержит функции для чтения и записи данных во флеш-память.
    \item \textbf{Main Module(Основная точка входа в программу)} \\
     Отвечает за инициализацию системы и подключение модулей. Также содержит основной цикл работы, в котором происходит вызов функций других модулей.
\end{enumerate}

\par \noindent \textbf{Основная логика работы программы}

Основная логика работы, инициализация и управление модулями содержится в Main Module программы. Ключевыми элементами данного модуля являются две основные функции Arduino Framework, а именно:

\begin{itemize}
    \item \textbf{setup():} Инициализация системы и подключение модулей.
    \item \textbf{loop():} Основной цикл работы, в котором происходит вызов функций других модулей.
\end{itemize}

Рассмотрим основную логику работы этих функций:

\par \textbf{setup():}

\begin{enumerate}
    \item Инициализация серийного порта для вывода отладочной информации.
    \item Настройка BLE-сервиса с помощью вызова соответствующих функций из BLE Service Handler.
    \item Загрузка конфигурации из SPI Flash памяти.
    \item Инициализация всех необходимых периферийных устройств и модулей, таких как сенсоры и Wi-Fi.
\end{enumerate}

\par \textbf{loop():}

\begin{enumerate}
    \item Циклический опрос состояния системы.
    \item Сбор данных с сенсоров и их обработка.
    \item Упаковка данных с помощью Data Package Module.
    \item Передача данных через BLE или по Wi-Fi на сервер.
    \item Управление состояниями системы и обработка событий, таких как потеря соединения или ошибки передачи данных. (В данном случае данные сохраняются на SD карту)
\end{enumerate}

\par \noindent \textbf{Тестирование}

В ходе разработки проекта возникли ограничения в виде отсутствия собранного тестирующего стенда, что затруднило проведение автоматизированного традиционного тестирования.

На данный момент применяется тестирование модуля только в области форматирования и проверки стиля кода, который на него загружается, что в конечном итоге все равно способствует повышению его читаемости и качества.

% ========== Application layer ==========
\subsubsection{Приложение для передачи данных (Application Layer)}

% ========== Server ==========
\subsubsection{Сервер для сбора, анализа и хранения данных (Server Layer)}

% ========== Data storage level ==========
\subsubsection{База данных (Data Storage Layer)}

% ========== Web layer ==========
\subsubsection{Web Layer (Веб-приложение)}

% ========== Conclusion ==========
\newpage
\section{Практическая апробация и внедрение}

% ========== References ==========
\newpage
\section{Список литературы}


% ========== Additional 
\newpage
\section{Приложение}

\end{spacing}
\end{document}



